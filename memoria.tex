% Options for packages loaded elsewhere
\PassOptionsToPackage{unicode}{hyperref}
\PassOptionsToPackage{hyphens}{url}
\PassOptionsToPackage{dvipsnames,svgnames,x11names}{xcolor}
%
\documentclass[
]{journal}

\usepackage{amsmath,amssymb}
\usepackage{iftex}
\ifPDFTeX
  \usepackage[T1]{fontenc}
  \usepackage[utf8]{inputenc}
  \usepackage{textcomp} % provide euro and other symbols
\else % if luatex or xetex
  \usepackage{unicode-math}
  \defaultfontfeatures{Scale=MatchLowercase}
  \defaultfontfeatures[\rmfamily]{Ligatures=TeX,Scale=1}
\fi
\usepackage{lmodern}
\ifPDFTeX\else  
    % xetex/luatex font selection
\fi
% Use upquote if available, for straight quotes in verbatim environments
\IfFileExists{upquote.sty}{\usepackage{upquote}}{}
\IfFileExists{microtype.sty}{% use microtype if available
  \usepackage[]{microtype}
  \UseMicrotypeSet[protrusion]{basicmath} % disable protrusion for tt fonts
}{}
\makeatletter
\@ifundefined{KOMAClassName}{% if non-KOMA class
  \IfFileExists{parskip.sty}{%
    \usepackage{parskip}
  }{% else
    \setlength{\parindent}{0pt}
    \setlength{\parskip}{6pt plus 2pt minus 1pt}}
}{% if KOMA class
  \KOMAoptions{parskip=half}}
\makeatother
\usepackage{xcolor}
\usepackage[top=30mm,left=20mm]{geometry}
\setlength{\emergencystretch}{3em} % prevent overfull lines
\setcounter{secnumdepth}{5}
% Make \paragraph and \subparagraph free-standing
\ifx\paragraph\undefined\else
  \let\oldparagraph\paragraph
  \renewcommand{\paragraph}[1]{\oldparagraph{#1}\mbox{}}
\fi
\ifx\subparagraph\undefined\else
  \let\oldsubparagraph\subparagraph
  \renewcommand{\subparagraph}[1]{\oldsubparagraph{#1}\mbox{}}
\fi


\providecommand{\tightlist}{%
  \setlength{\itemsep}{0pt}\setlength{\parskip}{0pt}}\usepackage{longtable,booktabs,array}
\usepackage{calc} % for calculating minipage widths
% Correct order of tables after \paragraph or \subparagraph
\usepackage{etoolbox}
\makeatletter
\patchcmd\longtable{\par}{\if@noskipsec\mbox{}\fi\par}{}{}
\makeatother
% Allow footnotes in longtable head/foot
\IfFileExists{footnotehyper.sty}{\usepackage{footnotehyper}}{\usepackage{footnote}}
\makesavenoteenv{longtable}
\usepackage{graphicx}
\makeatletter
\def\maxwidth{\ifdim\Gin@nat@width>\linewidth\linewidth\else\Gin@nat@width\fi}
\def\maxheight{\ifdim\Gin@nat@height>\textheight\textheight\else\Gin@nat@height\fi}
\makeatother
% Scale images if necessary, so that they will not overflow the page
% margins by default, and it is still possible to overwrite the defaults
% using explicit options in \includegraphics[width, height, ...]{}
\setkeys{Gin}{width=\maxwidth,height=\maxheight,keepaspectratio}
% Set default figure placement to htbp
\makeatletter
\def\fps@figure{htbp}
\makeatother

\makeatletter
\@ifpackageloaded{caption}{}{\usepackage{caption}}
\AtBeginDocument{%
\ifdefined\contentsname
  \renewcommand*\contentsname{Tabla de contenidos}
\else
  \newcommand\contentsname{Tabla de contenidos}
\fi
\ifdefined\listfigurename
  \renewcommand*\listfigurename{Listado de Figuras}
\else
  \newcommand\listfigurename{Listado de Figuras}
\fi
\ifdefined\listtablename
  \renewcommand*\listtablename{Listado de Tablas}
\else
  \newcommand\listtablename{Listado de Tablas}
\fi
\ifdefined\figurename
  \renewcommand*\figurename{Figura}
\else
  \newcommand\figurename{Figura}
\fi
\ifdefined\tablename
  \renewcommand*\tablename{Tabla}
\else
  \newcommand\tablename{Tabla}
\fi
}
\@ifpackageloaded{float}{}{\usepackage{float}}
\floatstyle{ruled}
\@ifundefined{c@chapter}{\newfloat{codelisting}{h}{lop}}{\newfloat{codelisting}{h}{lop}[chapter]}
\floatname{codelisting}{Listado}
\newcommand*\listoflistings{\listof{codelisting}{Listado de Listados}}
\makeatother
\makeatletter
\makeatother
\makeatletter
\@ifpackageloaded{caption}{}{\usepackage{caption}}
\@ifpackageloaded{subcaption}{}{\usepackage{subcaption}}
\makeatother
\ifLuaTeX
\usepackage[bidi=basic]{babel}
\else
\usepackage[bidi=default]{babel}
\fi
\babelprovide[main,import]{spanish}
% get rid of language-specific shorthands (see #6817):
\let\LanguageShortHands\languageshorthands
\def\languageshorthands#1{}
\ifLuaTeX
  \usepackage{selnolig}  % disable illegal ligatures
\fi
\usepackage{bookmark}

\IfFileExists{xurl.sty}{\usepackage{xurl}}{} % add URL line breaks if available
\urlstyle{same} % disable monospaced font for URLs
\hypersetup{
  pdftitle={Recomendación de anuncios},
  pdfauthor={Jorge Almonacid, Rubén Genillo, Humberto Pérez de la Blanca, Cristina Suárez; Universidad San Pablo CEU},
  pdflang={es},
  colorlinks=true,
  linkcolor={blue},
  filecolor={Maroon},
  citecolor={Blue},
  urlcolor={Blue},
  pdfcreator={LaTeX via pandoc}}

\title{Recomendación de anuncios}
\author{Jorge Almonacid, Rubén Genillo, Humberto Pérez de la Blanca,
Cristina Suárez \and Universidad San Pablo CEU}
\date{}

\begin{document}
\maketitle

\renewcommand*\contentsname{Indice}
{
\hypersetup{linkcolor=}
\setcounter{tocdepth}{3}
\tableofcontents
}
\newpage{}

\section{Brainstorming}\label{brainstorming}

Durante el brainstorming, se exploraron diversas áreas de interés, desde
la seguridad informática con el análisis de correos maliciosos y el
desarrollo de un detector de malware, hasta la aplicación de la teoría
de juegos para identificar posibles trampas, lo que representa un
enfoque innovador con amplias implicaciones. Además, se abordó el campo
de la inteligencia artificial con la creación de un modelo de generación
de texto, lo que refleja un interés en tecnologías avanzadas y su
impacto potencial.

Se seleccionó una idea centrada en el ámbito del análisis de datos en
redes sociales para la recomendación de anuncios. Esta propuesta
demuestra una comprensión perspicaz del papel cada vez más relevante que
desempeñan las opiniones y tendencias en línea en el mundo del marketing
digital.

\subsubsection{Necesidades:}\label{necesidades}

Al haber seleccionado la idea centrada en el análisis de datos en redes
sociales para la recomendación de anuncios, se pone de manifiesto la
importancia de estar actualizados, ser precisos y comprender las
respuestas de los clientes. Estas necesidades reflejan la relevancia de
mantenerse al tanto de las tendencias cambiantes en línea, la precisión
en la interpretación de datos y la comprensión profunda de las
interacciones de los clientes en entornos digitales. Este enfoque
resalta la importancia estratégica del marketing en un mundo cada vez
más impulsado por la información y las interacciones en línea.

\section{Objetivo}\label{objetivo}

El objetivo de la extracción de datos y opiniones en redes sociales de
usuarios para la recomendación de anuncios se fundamenta en la necesidad
de acceder a información pública, en tiempo real y constantemente
actualizada. Este enfoque estratégico busca aprovechar la riqueza de
datos disponibles en entornos digitales públicos, garantizar la
relevancia y actualidad de la información recopilada, y capturar las
tendencias y opiniones más recientes. Al hacerlo, se busca informar
decisiones de marketing con datos dinámicos y significativos, alineados
con las demandas cambiantes del mercado y las interacciones de los
usuarios en línea.

Además, este enfoque puede tener usos alternativos significativos que
van más allá de la recomendación de anuncios. Por ejemplo, la extracción
de datos y opiniones en redes sociales puede ser invaluable para la
investigación de clientes, permitiendo una comprensión más profunda de
sus necesidades, preferencias y comportamientos en línea. Asimismo, la
identificación de \emph{Brand Ambassadors}, es decir, usuarios
influyentes que puedan promover de manera auténtica una marca, es otro
beneficio clave de esta estrategia. Además, el manejo de reputación en
línea se ve reforzado por la capacidad de monitorear y responder de
manera proactiva a las interacciones en redes sociales, lo que puede
impactar positivamente la percepción pública de una empresa o producto.
Estos son solo algunos ejemplos de cómo este enfoque puede ser
aprovechado para una variedad de aplicaciones estratégicas más allá de
la publicidad y recomendaciones comerciales.

\section{Tareas}\label{tareas}

La tarea se puede dividir en los siguientes temas:

\subsubsection{Extracción de Datos:}\label{extracciuxf3n-de-datos}

En esta fase, se enfocará en la extracción de datos clave de la
plataforma Twitter. Se buscará obtener información pública de los
usuarios, como nombres, biografías, publicaciones, día y hora de
publicación, comentarios, interacciones como me gusta, seguidores,
seguidos y hashtags utilizados. Esta información será recopilada a
través de una API que permitirá acceder a estos datos de manera
estructurada. Además, se profundizará en la obtención de información
personal más detallada, como edad, género, localización, ocupación e
intereses de los usuarios para enriquecer el análisis.

\subsubsection{Filtrar posts
relevantes:}\label{filtrar-posts-relevantes}

Una vez recopilados los datos, el siguiente paso será filtrar y
clasificar los posts relevantes. Para lograr esto, se implementará un
análisis de temas avanzado. Se utilizarán algoritmos especializados como
el Análisis Semántico Latente (LSA) y la Asignación Latente de Dirichlet
(LDA) para identificar patrones y agrupar los posts según los temas
principales que abordan. Este proceso permitirá segmentar la información
de manera efectiva y comprender mejor las discusiones que tienen lugar
en la plataforma.

\subsubsection{Análisis de
sentimiento:}\label{anuxe1lisis-de-sentimiento}

Otro aspecto fundamental de esta tarea es el análisis de sentimiento.
Aquí se enfocará en determinar la polaridad de las opiniones expresadas
en los posts recopilados. Para ello, se emplearán lexicons de
sentimiento que ayudarán a clasificar las opiniones como positivas,
negativas o neutras. Además, se explorará un método alternativo que
involucra el uso de LMQL o langchain con Large Language Models para
mejorar la precisión y profundidad del análisis de sentimiento.

Se estudiarán detenidamente los datos obtenidos de Twitter para llevar a
cabo estas tareas con rigor y obtener insights significativos que puedan
impulsar estrategias efectivas en marketing digital y toma de decisiones
empresariales.

\section{Diagramas}\label{diagramas}

\section{Proyecto}\label{proyecto}



\end{document}
